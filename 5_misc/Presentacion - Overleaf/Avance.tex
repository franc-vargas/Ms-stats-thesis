% Copyright 2019 Clara Eleonore Pavillet

% Author: Clara Eleonore Pavillet
% Description: This is an unofficial Oxford University Beamer Template I made from scratch. Feel free to use it, modify it, share it.
% Version: 1.0

\documentclass[pdftex,spanish]{beamer}
\input{Theme/Packages.tex}
\usetheme{oxonian}
\usepackage{graphicx}
\usepackage{float}
%\floatstyle{boxed}
%\restylefloat{figure}
\usepackage{subfigure}
\usepackage{color}

% Math packages
\usepackage{amsmath}
\usepackage{amsfonts}
\usepackage{amssymb}

% Closest font to Times New Roman
\usepackage{times}

% To make pretty tables
\usepackage{booktabs}
\usepackage{multirow}

% To avoid underfull errors in the bibliography
\usepackage{etoolbox}
\apptocmd{\sloppy}{\hbadness 10000\relax}{}{}

% To make cites and references
\usepackage[hidelinks,pdfusetitle,pdfdisplaydoctitle]{}
\usepackage[notocbib]{apacite} 
\usepackage{doi}
\renewcommand{\doitext}{}

\title{Kriging Ordinario: Comparación utilizando distancias euclidianas y no euclidianas aplicadas a la salmonicultura}
\author{Francisco Vega}
\institute{Pontificia Universidad Cat\'olica de Chile}
\date{\today}

\titlegraphic{\includegraphics[width=2cm]{LogoUC (1).pdf}}

\begin{document}

{\setbeamertemplate{footline}{} 
\frame{\titlepage}}

\section*{Contenido}\begin{frame}{Contenido}\tableofcontents\end{frame}

\section{Proyecto de Magíster}
    \begin{frame}[plain]
        \vfill
      \centering
      \begin{beamercolorbox}[sep=8pt,center,shadow=true,rounded=true]{title}
        \usebeamerfont{title}\insertsectionhead\par%
        \color{oxfordblue}\noindent\rule{10cm}{1pt} \\
        \LARGE{\faFileTextO}
      \end{beamercolorbox}
      \vfill
  \end{frame}


\subsection{Objetivos}
\subsubsection{Generales}

\begin{frame}{Objetivos}
\noindent\textbf{Objetivo General.} \\
\indent Estimar la funci\'on de covarianza y variogramas en modelos geoestad\'isticos apliados a grandes conjuntos de datos utilizando una modificaci\'on de la metodolog\'ia propuesta por \citeA{Tucci2019}, y su comparaci\'on con respecto a metodolog\'ias tradicionales de estimaci\'on frente a datos no-experimentales.\\
\

\vfill
%\noindent {\bf Keywords}:  thesis template, document writing, {\bf (Write here the keywords relevant and strictly related to the topic of the thesis)}.


\end{frame}

\subsubsection{Espec\'ificos}

\begin{frame}{Objetivos}
\noindent\textbf{Objetivos Espec\'ificos}
\begin{enumerate}
    \item Comparar la estimaci\'on de la matriz de correlaci\'on de un modelo param\'etrico utilizando una metodolog\'ia alternativa adaptada a procesos espaciales con respecto a otros modelos cl\'asicos.
    \item Generar un algoritmo para estimaci\'on de par\'ametros de la funci\'on de covarianza para la funci\'on objetivo, cuando esta es casi-veros\'imil o veros\'imil-penalizada.
    \item Utilizar los modelos y algoritmo generado para evaluar el desempeño de los modelos frente a la presencia de datos obtenidos en un ambiente no-experimental.
    \item Utilizar los modelos y algoritmo generado para evaluar el desempeño de los modelos utilizando m\'etodo de Monte Carlo.
\end{enumerate}


\end{frame}

\section{Marco Te\'orico}
    \begin{frame}[plain]
        \vfill
      \centering
      \begin{beamercolorbox}[sep=8pt,center,shadow=true,rounded=true]{title}
        \usebeamerfont{title}\insertsectionhead\par%
        \color{oxfordblue}\noindent\rule{10cm}{1pt} \\
        \LARGE{\faFileTextO}
      \end{beamercolorbox}
      \vfill
  \end{frame}

\subsection{Conceptos}



\subsubsection{Proceso Estacionario}
\begin{frame}{Proceso Estacionario}

Un campo aleatorio, denotado por ${\{Y(\mathbf{s}) : \mathbf{s} \in D \subset \mathbb{R}^d\}}$ o ${\{Y(\mathbf{s})\}}$, es una colecci\'on de variables aleatorias indexadas por $D$, donde cada $\mathbf{s} \in D$ es una localizaci\'on geogr\'afica \cite{Cressie1993}.

\vspace{\baselineskip}
\begin{enumerate}
        \item Estrictamente estacionario: $Y(\mathbf{s})$, $\mu(\mathbf{s}) = \mathbb{E}(Y(\mathbf{s}))$  es una funci\'on constante de $\mathbf{s}$ \cite{Banerjee2015}.
        \item D\'ebilmente estacionario: $\mu(\mathbf{s}), \sigma^{2}(\mathbf{s})$ son constantes \cite{Schaenberger2005}.
        \item Intr\'insecamente estacionario: Es $Var(Y(\mathbf{s+h}) - Y(\mathbf{s}))$, libre de localizaci\'on y depende solo del vector $\mathbf{h}$ \cite{Sujit2022}.
\end{enumerate}

\end{frame}
\subsubsection{Variograma e Isotrop\'ia}
\begin{frame}{Variograma}
Aplica bajo el concepto de media y varianza constante, para un  proceso intrinsecamente estacionario podemos asumir: \vspace{\baselineskip}
\begin{itemize}
    \item $\mathbb{E}[Y\mathbf{(s + h)} - Y\mathbf{(s)}] = 0$
    \item $\mathbb{E}[Y\mathbf{(s + h)} - Y(\mathbf{s})]^2 = Var(Y\mathbf{(s+h)} - Y\mathbf{(s)}) = 2\gamma(\mathbf{h})$
\end{itemize}
\vspace{\baselineskip}
\subsubsection{Variograma}

donde el lado izquierdo depende solo de $\mathbf{h}$ y no de $\mathbf{s}$, y la funci\'on $2\gamma(\mathbf{h})$ es el variograma.\vspace{\baselineskip}

Y donde el gr\'afico de la covarianza frente a h $\mathbf{C(h)}$ es comunmente llamada covariograma.

\end{frame}
\subsubsection{Isotrop\'ia y Kriging}
\begin{frame}{Isotrop\'ia}
\textbf{Isotrop\'ia}\\
Si es que la funci\'on del semivariograma depende de la funci\'on $\gamma(\mathbf{h})$, el variograma es isotr\'opico, sino se define como anisotr\'opico\vspace{\baselineskip}

\textbf{Kriging}\\
Permite estimar el valor de una variable sobre un campo aleatorio espacial continuo, basado en la autocorrelaci\'on y ponderaciones.
\begin{equation}
   \mathbf{\hat{Z}(x_0)} = \sum_{i=1}^{n}\lambda_i
\end{equation}
donde $\sum_{i=1}^n\lambda_i = 1$ \cite{FischerHandbook2010}.
\end{frame}
\subsubsection{Modelos}
\begin{frame}{Modelos Param\'etricos}
Tienen como base los modelos lineales cl\'asicos, adaptados al punto de vista espacial, donde la correlaci\'on se representa a trav\'es  de $\mathbf{h}$ \cite{Dormann2007}. 
\begin{itemize}
    \item Autocovariados $\mathbf{Y} = \mathbf{X} \boldsymbol{ \beta } + \boldsymbol{\varepsilon} \rightarrow \mathbf{Y}=\mathbf{X}\boldsymbol{ \beta }+\rho A+\boldsymbol{\varepsilon}$
    \item M\'inimos Cuadrados $\mathbf{Y} = \mathbf{X} \boldsymbol{ \beta } + \boldsymbol{\varepsilon}$ con $ \boldsymbol{\varepsilon} \sim N (0,\boldsymbol{\Sigma})$
    \begin{itemize}
    \item Exponencial.
    \item Gaussiana.
    \item Esf\'erica.
    \end{itemize}
    \item Aproximaciones a los GLM.
    \begin{itemize}
        \item Ej. Autolog\'istico \cite{Dormann2007}
    \end{itemize}
    
\end{itemize}

\end{frame}

\subsubsection{Evaluaci\'on de los modelos}
\begin{frame}{Evaluaci\'on modelos}
Se considerar\'a una metodolog\'ia basada \cite{Tucci2019}, quienes proponen un m\'etodo nuevo para estimar la matriz de covarianza real $\Sigma$ desde la matriz de covarianza no-negativa definitiva $m \times m$ de la muestra, definida como $K = \frac{1}{n}MM^*$ bajo la condici\'on de $n<m$.\vspace{\baselineskip}

Este nuevo modelo param\'etrico ser\'a contrastado con un modelo param\'etrico cl\'asico y el desempeño de ambos ser\'a evaluado a trav\'es de su comportamiento frente a datos no-experimentales (de campo) y tambi\'en con m\'etodo de Monte Carlo.

\end{frame}


% \section{Equations}
%     \begin{frame}[plain]
%         \vfill
%       \centering
%       \begin{beamercolorbox}[sep=8pt,center,shadow=true,rounded=true]{title}
%         \usebeamerfont{title}\insertsectionhead\par%
%         \color{oxfordblue}\noindent\rule{10cm}{1pt} \\
%         \LARGE{\faFileTextO}
%       \end{beamercolorbox}
%       \vfill
%   \end{frame}
  
% \subsection{Example}
% \begin{frame}{Example}
% \only<1>{
% Let \(p(x)=\mathcal{N}(\mu\textsubscript{1},\,\sigma^{2}\textsubscript{1})\) and \(q(x)=\mathcal{N}(\mu\textsubscript{2},\,\sigma^{2}\textsubscript{2})\): \\
% \begin{equation}
% \mathcal{N}=\frac{1}{\sigma\,\sqrt{2\,\pi}}\,\E^{-\frac{\left(x-\mu\right)^2}{2\,\sigma^2} }
% \end{equation}}
% \only<2>{
% Kullback-Leibler divergence for continuous probabilities:
% \begin{align*}
% 	D(p,q)=&\int p(x) \log \frac{p(x)}{q(x)}\ud x\\
%     =& \int p(x) \,\ln p(x) \ud x -\int p(x) \,\ln q(x) \ud x\\
% 	=&\,\frac{1}{2} \ln\left(2\,\pi\,\sigma_2^{2}\right) +\frac{\sigma_1^{2}+\left(\mu_1-\mu_2 \right)^2 }{2\,\sigma_2^2}-\frac{1}{2}\left( 1+\ln 2\,\pi\,\sigma_1^2\right) \\
% 	=&\,\ln\frac{\sigma_2}{\sigma_1} +\frac{\sigma_1^{2}+\left(\mu_1-\mu_2 \right)^2 }{2\,\sigma_2^2}-\frac{1}{2}
% \end{align*}
% }
% \end{frame}

% \section{Code}
%     \begin{frame}[plain]
%         \vfill
%       \centering
%       \begin{beamercolorbox}[sep=8pt,center,shadow=true,rounded=true]{title}
%         \usebeamerfont{title}\insertsectionhead\par%
%         \color{oxfordblue}\noindent\rule{10cm}{1pt} \\
%         \LARGE{\faFileCodeO}
%       \end{beamercolorbox}
%       \vfill
%   \end{frame}
  
% \subsection{Example}
% \begin{frame}[fragile]{Example}
% \begin{block}{Greatest Common Divisor}
% \begin{lstlisting}[firstnumber=1, label=glabels, xleftmargin=10pt] 
% def greatest_c_remainder(a,b):
% 	'''Greatest common divisor of a and b'''
% 	r = a % b
% 	if r == 0:
% 		return b
% 	else:
% 		m = b
% 		n = r
% 	return greatest_c_remainder(m,n)

% \end{lstlisting}
% \end{block}
% \end{frame}

\section{REFERENCIAS}
\begin{frame}[allowframebreaks]{REFERENCIAS}
\bibliographystyle{apacite}
\renewcommand{\bibname}{REFERENCIAS}
\bibliography{References} 
\end{frame}


\end{document}

